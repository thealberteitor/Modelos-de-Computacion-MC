\input{preambuloSimple.tex}  % Configuración del documento

%----------------------------------------------------------------------------------------
%	TÍTULO Y DATOS DE LOS ALUMNOS
%----------------------------------------------------------------------------------------

\title{	
	\normalfont \normalsize 
	\textsc{\textbf{Modelos de Computación (2017-2018)} \\ Doble Grado en Ingeniería Informática y Matemáticas \\ Universidad de Granada} \\ [25pt] 
	\horrule{0.5pt} \\[0.4cm]
	\huge PETICIÓN HTTP - LEX \\ 
	\horrule{2pt} \\[0.5cm] 
}

\author{ Alberto Jesús Durán López} 
\date{\normalsize\today}

%----------------------------------------------------------------------------------------
% DOCUMENTO
%----------------------------------------------------------------------------------------

\begin{document}
	\maketitle       % título
	\newpage 
	\tableofcontents % índice
	\newpage
	
	
	
	
	\section{Introducción}
	
	Lex es una herramienta que nos permite ejecutar acciones tras la localización de cadenas de entrada que emparejan con expresiones regulares.
	
	En esta práctica, he realizado un programa en el que podemos comprobar si una petición HTTP es correcta. 	
	
	
	\section{Petición HTTP}
	
	Antes de comenzar con la explicación del programa, introduciremos su funcionamiento.
	A simple vista, el proceso de acceder una página web y navegar por ella es sencillo, sin embargo, resulta un proceso algo más complejo ya que se usan una serie de protocolos que hacen posible el acceso a la web requerida. 
	Cuando un usuario desea iniciar una conexión  al servidor HTTP, deberá enviar una petición en el que indique que acción desea hacer. 
	
	Una petición HTTP está formada por una línea de solicitud, que incluye el método y la URL , y una serie de líneas de cabecera.
	
	\begin{itemize}
		\item \textbf{Método:}
		\begin{itemize}
			\item GET: solicitud del un recurso.
			\item HEAD: similar a GET pero sólo se devuelve la cabecera.
			\item PUT: solicitud de sustituir la URL especificada con los datos incluídos.
			\item POST: solicitid de acptar la URL especificada con los datos incluídos.
			\item DELETE: solicitud de borrar la url especificada.
			\item OPTIONS: solicitud de información.
		\end{itemize}
		
		\item \textbf{Dirección-path}: /eltiempo/granada.html
		\item \textbf{Host}: www.google.es / www.ugr.es
		\item \textbf{Navegador usado}: Mozilla, chrome, safari...etc
	\end{itemize}
	
	Además, en la petición se incluyen los datos, que al ser independientes en cada consulta no los hemos añadido.
	
	Acto seguido, el servidor HTTP contestará al cliente con un mensaje de éxito (enviando el objeto encapsulado) o con un mensaje de error (enviando su correspondiente código de error).
	
	
	
	\newpage
	\section{Programa HTTP interactivo}
	
	Para cada campo de una petición HTTP (ya comentados anteriormente), he creado una expresión regular capaz de reconocerlos. Además, he añadido algunas expresiones regulares más para acceder a un ejemplo dado o para buscar información en google que pueda servir de ayuda.
	
	Una vez hayamos compilado y ejecutado el programa, se abrirá el menú siguiente:
	
	\begin{figure}[h]
		\centering
		\includegraphics[width=.8\textwidth]{1}
		\caption{Inicio del programa}
	\end{figure}
	
	Como podemos observar, simplemente bastará con pulsar la tecla '0', '1' o introducir la dirección 'www.google.es' en la terminal para acceder a la opción deseada. \\
	
	
	Una vez hayamos introducido todos los campos correctamente, se mostrará el mensaje siguiente:
	
	\begin{figure}[h]
		\centering
		\includegraphics[width=.8\textwidth]{2}
		\caption{Petición HTTP final}
	\end{figure}
	
	Cabe destacar, que el programa no hace nada si se introduce algo que no coincida con alguna expresión regular y si introducimos una frase cuya expresión regular coincida con alguna que ya se ha introducido, se mostrará un mensaje de error.
	
	
\end{document}