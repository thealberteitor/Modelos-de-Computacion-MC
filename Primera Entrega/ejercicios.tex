%%%%%%%%%%%%%%%%%%%%%%%%%%%%%%%%%%%%%%%%%
% Short Sectioned Assignment LaTeX Template Version 1.0 (5/5/12)
% This template has been downloaded from: http://www.LaTeXTemplates.com
% Original author:  Frits Wenneker (http://www.howtotex.com)
% License: CC BY-NC-SA 3.0 (http://creativecommons.org/licenses/by-nc-sa/3.0/)
%%%%%%%%%%%%%%%%%%%%%%%%%%%%%%%%%%%%%%%%%

%----------------------------------------------------------------------------------------
%	PACKAGES AND OTHER DOCUMENT CONFIGURATIONS
%----------------------------------------------------------------------------------------

\documentclass[paper=a4, fontsize=11pt]{scrartcl} % A4 paper and 11pt font size

% ---- Entrada y salida de texto -----

\usepackage[T1]{fontenc} % Use 8-bit encoding that has 256 glyphs
\usepackage[utf8]{inputenc}
%\usepackage{fourier} % Use the Adobe Utopia font for the document - comment this line to return to the LaTeX default

% ---- Idioma --------

\usepackage[spanish, es-tabla]{babel} % Selecciona el español para palabras introducidas automáticamente, p.ej. "septiembre" en la fecha y especifica que se use la palabra Tabla en vez de Cuadro

% ---- Otros paquetes ----

\usepackage{url} % ,href} %para incluir URLs e hipervínculos dentro del texto (aunque hay que instalar href)
\usepackage{amsmath,amsfonts,amsthm} % Math packages
%\usepackage{graphics,graphicx, floatrow} %para incluir imágenes y notas en las imágenes
\usepackage{graphics,graphicx, float} %para incluir imágenes y colocarlas

% Para hacer tablas comlejas
%\usepackage{multirow}
%\usepackage{threeparttable}

%\usepackage{sectsty} % Allows customizing section commands
%\allsectionsfont{\centering \normalfont\scshape} % Make all sections centered, the default font and small caps

\usepackage{fancyhdr} % Custom headers and footers
\pagestyle{fancyplain} % Makes all pages in the document conform to the custom headers and footers
\usepackage{eurosym} % Para poder añadir el símbolo del euro
\fancyhead{} % No page header - if you want one, create it in the same way as the footers below
\fancyfoot[L]{} % Empty left footer
\fancyfoot[C]{} % Empty center footer
\fancyfoot[R]{\thepage} % Page numbering for right footer
\renewcommand{\headrulewidth}{0pt} % Remove header underlines
\renewcommand{\footrulewidth}{0pt} % Remove footer underlines
\setlength{\headheight}{13.6pt} % Customize the height of the header

\numberwithin{equation}{section} % Number equations within sections (i.e. 1.1, 1.2, 2.1, 2.2 instead of 1, 2, 3, 4)
\numberwithin{figure}{section} % Number figures within sections (i.e. 1.1, 1.2, 2.1, 2.2 instead of 1, 2, 3, 4)
\numberwithin{table}{section} % Number tables within sections (i.e. 1.1, 1.2, 2.1, 2.2 instead of 1, 2, 3, 4)

\setlength\parindent{0pt} % Removes all indentation from paragraphs - comment this line for an assignment with lots of text

\newcommand{\horrule}[1]{\rule{\linewidth}{#1}} % Create horizontal rule command with 1 argument of height


%----------------------------------------------------------------------------------------
%	TÍTULO Y DATOS DEL ALUMNO
%----------------------------------------------------------------------------------------

\title{	
\normalfont \normalsize 
\textsc{\textbf{Modelos de computación (2017-2018)} \\ Doble Grado en Ingeniería Informática y Matemáticas \\ Universidad de Granada} \\ [25pt] % Your university, school and/or department name(s)
\horrule{0.5pt} \\[0.4cm] % Thin top horizontal rule
\huge Relación de problemas I \\ % The assignment title
\horrule{2pt} \\[0.5cm] % Thick bottom horizontal rule
}

\author{Alberto Jesús Durán López} % Nombre y apellidos

\date{\normalsize\today} % Incluye la fecha actual

%----------------------------------------------------------------------------------------
% DOCUMENTO
%----------------------------------------------------------------------------------------

\begin{document}

\maketitle % Muestra el Título

\newpage %inserta un salto de página

\tableofcontents % para generar el índice de contenidos

%\listoffigures

% \listoftables

\newpage













\section{Ejercicio 13}
\begin{itemize}
	\item \textbf{Dados dos homomorfismos f: A* $\rightarrow$ B* , g: A* $\rightarrow$ B*, se dice que son iguales si f(x)=g(x), $\forall$x $\in$ A*. ¿Existe un procedimiento algorítmico para comprobar si dos homomorfismos son iguales?}
\end{itemize} 

Nota: 
Si x es un elemento cualquiera de un grupo, f es un homomorfismo y n es un entero mayor o igual que cero, por la definición de homomorfismo tenemos que: \[ f(x^n) = f(x)^n \] \\
Tampoco es difícil ver que: \[ f(x^{-1})=f(x)^{-1} \] \\
Si n es un entero negativo, entonces -n es un entero positivo por lo que aplicando el resultado anterior, se tiene \[ f(x^{n})=f((x^{-1})^{-n})=f(x^{-1})^{-n}=f(x)^{(-1)(-n)}=f(x)^n\] \\ 
En consecuencia, \[ f(x^{n})=f(x)^{n} , \forall n \in \mathbb{Z} \] \\


\horrule{0.5pt} 


Sabemos que para que  f: G $\rightarrow$ {G'} y  g: G$\rightarrow${G'} sean iguales, $\forall${y}$\in${G} ,  f(y)=g(y). 
Tenemos que si y $\in$ {G}  entonces:
\[y=x_1^{k_1}\cdot ... \cdot x_n^{k_n}\] para algunos \[k_1,...,k_n \in \mathbb{Z}\] \\
Esto nos garantiza que: 
\begin{center}
	\[f(y)=f(x_1^{k_1}\cdot ... \cdot x_n^{k_n})=f(x_1^{k_1})\cdot ...\cdot  f(x_n^{k_n})= \\ =f(x_1)^{k_1}\cdot ...\cdot f(x_n)^{k_n}= g(x_1)^{k_1}\cdot ...\cdot g(x_n)^{k_n} =  g(x_1^{k_1})\cdot ...\cdot g(x_n^{k_n}) = g(x_1^{k_1}\cdot ... \cdot x_n^{k_n}) = g(y)\].
	
\end{center}





\newpage
\section{Ejercicio 16}

\begin{itemize}
	\item \textbf{Dada la gramática \\
		G=(\{S,A\} , \{a,b\} , P, S $\rightarrow$ abAS , abA $\rightarrow$ baab , S $\rightarrow$ a , A $\rightarrow$ b).
		 \\ Determinar el lenguaje que genera.} 
\end{itemize}

Observando las reglas de producción de P, podemos darnos cuenta que siempre vamos a llegar a una palabra terminada en 'a', es decir: \\

S $\rightarrow$ a \\
S $\rightarrow$ abAS $\rightarrow$ abba \\
S $\rightarrow$ abAa $\rightarrow$ baaba \\
S $\rightarrow$ abAabAS $\rightarrow$ ... $\rightarrow$ ... a \\ \\

Por tanto: 
\begin{center}
	
	L = $\left\lbrace u \cdot a: u \in A^* \right\rbrace$
\end{center}




\newpage
\section{Ejercicio 17}
\begin{itemize}
	\item \textbf{Sea la gramática G = (V,T,P,S) donde: \\
		-V = \{<número>, <dígito>\} \\
		-T = \{0,1,2,3,4,5,6,7,8,9\} \\
		-S = < número> \\
		-Las reglas de producción de P son:
		\begin{center}
			<número> $\rightarrow$ <número><dígito> \\
			<número> $\rightarrow$ <dígito> \\
			<dígito> $\rightarrow$ 0|1|2|3|4|5|6|7|8|9
		\end{center}
		Determinar el lenguaje que genera.
	}	
\end{itemize}

Repasaremos unas nociones básicas antes de seguir con el ejercicio: \\

-V es un alfabeto, llamado variables o símbolos no terminales. Sus elementos se representan con letras mayúsculas. \\
-T son los símbolos terminales, se representan con letras minúsculas \\
S es el símbolo de partida \\

Además, hacemos un cambio de notación: \\
\begin{center}
	S = <número> \\
	A = <dígito> \\
	a $\in \left[0,9\right]$
\end{center}

Resultando: \\
\begin{center}
	S $\rightarrow$ SA \\
	S $\rightarrow$ A \\
	A $\rightarrow$ 0|1|2|3|4|5|6|7|8|9 \\
\end{center}

Podemos observar que: \\

S $\rightarrow$ SA $\rightarrow$ SAA $\rightarrow$ ... $\rightarrow$ ... aaaa \\
S $\rightarrow$ SA $\rightarrow$ SAA $\rightarrow$ Aaa $\rightarrow$ aaa \\
S $\rightarrow$ A $\rightarrow$ a \\

Por tanto, todas las palabras que se pueden formar están formados por números a partir del 0 en adelante, es decir, el lenguaje generado por esta gramática es:

\begin{center}
	
	L = $\left\lbrace n: n \in \mathbb{Z}^+ \right\rbrace$
\end{center}





\newpage
\section{Ejercicio 18}
\begin{itemize}
	\item \textbf{Sea la gramática G=($\left\lbrace A,S\right\rbrace$, $\left\lbrace a,b \right\rbrace$, S, P) donde las reglas de producción son:
	\begin{center}
		S $\rightarrow$ aS \\ S $\rightarrow$ aA \\ A $\rightarrow$ bA \\A $\rightarrow$ b
	\end{center}
	Determinar el lenguaje que genera.
	 }
\end{itemize}

Observando las reglas de producción de la gramática, vemos que cada vez que tenemos una A, obtenemos una b por lo que siempre vamos a obtener palabras terminadas en 'b', es decir: \\

S $\rightarrow$ aS $\rightarrow$ aaS $\rightarrow$ aaaS $\rightarrow$ ... $\rightarrow$ ... b \\
S $\rightarrow$ aS $\rightarrow$ aaA $\rightarrow$ aabA $\rightarrow$ ... $\rightarrow$ ... b \\
S $\rightarrow$ aS $\rightarrow$ aaA $\rightarrow$ aab\\
S $\rightarrow$ aA $\rightarrow$ ab\\
S $\rightarrow$ aA $\rightarrow$ abbA $\rightarrow$ ... b\\
S $\rightarrow$ aA $\rightarrow$ abA $\rightarrow$ abb\\

Por tanto: 
\begin{center}
	
	L = $\left\lbrace u \cdot b: u \in A^* \right\rbrace$
\end{center}




\newpage
\section{Ejercicio 19}
\begin{itemize}
	\item \textbf{Encontrar si es posible una gramática lineal por la derecha o una gramática independiente del contexto que genere el lenguaje L, en cada uno de los casos, supuesto que L $\subseteq$ $\left\lbrace{a,b,c}\right\rbrace$* y verifica: \\ \\
	a) u $\in$ L sii verifica que u no contiene dos símbolos consecutivos \\
	b) u $\in$ L sii verifica que u contiene dos símbolos b consecutivos \\
	c) u $\in$ L sii verifica que contiene un número impar de símbolos c \\
	d) u $\in$ L sii verifica que no contiene el mismo número de símbolos b que de símbolos c}
\end{itemize}

\begin{center}
	a) Si se puede: \\
	S $\rightarrow$ $\epsilon$ \\
	S $\rightarrow$ cS \\
	S $\rightarrow$ aS\\
	S $\rightarrow$ bX\\
	X $\rightarrow$ aS\\
	X $\rightarrow$ $\epsilon$ \\
	X $\rightarrow$ cS \\
\end{center} 

\begin{center}
	b) Si se puede: \\
	S $\rightarrow$ aS \\
	S $\rightarrow$ bS \\
	S $\rightarrow$ cS \\
	S $\rightarrow$ bX \\
	X $\rightarrow$ bY \\
	Y $\rightarrow$ aY \\
	Y $\rightarrow$ bY \\
	Y $\rightarrow$ cY \\
	Y $\rightarrow$ $\epsilon$ \\
\end{center}

\begin{center}
	c) Si se puede: \\
	S $\rightarrow$ aS \\
	S $\rightarrow$ bS \\
	S $\rightarrow$ cX\\
	X $\rightarrow$ aX\\
	X $\rightarrow$ bX\\
	X $\rightarrow$ $\epsilon$ \\
	X $\rightarrow$ cS \\
\end{center}

\begin{center}
	d) No se puede \\
	
\end{center}








\end{document}